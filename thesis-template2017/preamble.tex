%!TEX encoding = UTF-8 Unicode
%
% 卒論 / 修論用 プリアンブル
%

%% フォント
\usepackage{lmodern}
\usepackage[scale=0.95]{tgheros}
\usepackage{textcomp}
\usepackage[scaled=0.85]{beramono}
\usepackage[T1]{fontenc}

%% パッケージ
\usepackage[cmex10]{amsmath}
\usepackage{amssymb,amsfonts,mathtools,bm}

\usepackage{graphicx,color}
\usepackage[table]{xcolor}

\usepackage[pdfborder={0 0 0}]{hyperref}
\usepackage{pxjahyper}

% caption のコロン「図4.4: キャプション」を「図4.4 キャプション」に直す
\usepackage[labelsep=quad,compatibility=false]{caption}
\usepackage[belowskip=1.2em]{subcaption}

\usepackage{cite,url,array,makecell}
\usepackage{algorithm,algorithmic}

% 数学コマンドの補完
\DeclareMathOperator*{\sinc}{sinc}
\DeclareMathOperator*{\prox}{prox}
\DeclareMathOperator*{\argmin}{argmin}
\DeclareMathOperator*{\argmax}{argmax}

% 参考文献表示スタイルを変更
\bibliographystyle{sieicej}

% 赤色を少し暗くする
\definecolor{red}{rgb}{0.75,0,0}

\makeatletter
   % アルゴリズム図キャプションの表記を「Algorithm」→「アルゴリズム」に
   \renewcommand{\ALG@name}{アルゴリズム}
\makeatother

% 修正箇所に色付けするコマンド
%
% ・ コマンド版
%   \fixed{修正箇所}
%
% ・ 環境版
%   \begin{fixedregion}
%      修正箇所
%   \end{fixedregion}
\newcommand{\fixed}[1]{#1} 
\newenvironment{fixedregion}{\ignorespaces}{\ignorespacesafterend}
% 下の 2 行をコメントアウトすることで色付けを無効化します
\renewcommand{\fixed}[1]{\textcolor{red}{#1}}
\renewenvironment{fixedregion}{\protect\leavevmode\color{red}\ignorespaces}{\ignorespacesafterend}

% 強調
\newcommand{\strong}[1]{\textcolor{red}{\textbf{#1}}}

