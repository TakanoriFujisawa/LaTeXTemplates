%!TEX root = manuscript.tex
%%% IEEE 系学会向け LaTeX テンプレート (プリアンブル)

% フォント関連
\usepackage[T1]{fontenc}
\usepackage{lmodern,newtxtext}
\normalfont % fd ファイルを読み込ませる
\DeclareFontShape{T1}{\rmdefault}{m}{sc}{<-> ptmrc8t}{}
\DeclareFontShape{T1}{\rmdefault}{b}{sc}{<-> ptmbc8t}{}
\DeclareFontShape{T1}{\rmdefault}{bx}{sc}{<-> ptmbc8t}{}

\usepackage{amsmath,amssymb,amsfonts,bm,mathtools}

% 日本語が入らないようにする
\usepackage[ascii]{inputenc}

\ifx\pfmtname\undefined
   \usepackage{graphicx,color,xcolor}
\else
   \usepackage[dvipdfmx]{graphicx,color,xcolor}
   % 日本語フォントの定義がなく,Warning が出るのでその対処
   % 1. IEEEtran のオプションに nofonttune を入れる (単語間スペース変更などの処理を無効に)
   % 2. 以下の日本語フォントを定義
   % 3. \CLASSOPTIONnofonttune を false にすると \begin{document} 時に fonttune してくれる
   %    ... のだがデフォルトの方が見やすいような気がするのでそのまま
   \DeclareFontShape{JY1}{mc}{m}{n}{<-> s*[0.961] jis}{}
   \DeclareFontShape{JY1}{gt}{m}{n}{<-> s*[0.961] jisg}{}
   \DeclareFontShape{JT1}{mc}{m}{n}{<-> s*[0.961] tmin10}{}
   \DeclareFontShape{JT1}{gt}{m}{n}{<-> s*[0.961] tgoth10}{}
   \DeclareFontShape{JY1}{mc}{m}{it}{<->ssub*mc/m/n}{}
   \DeclareFontShape{JY1}{mc}{m}{sl}{<->ssub*mc/m/n}{}
   \DeclareFontShape{JY1}{mc}{m}{sc}{<->ssub*mc/m/n}{}
   \DeclareFontShape{JY1}{gt}{m}{it}{<->ssub*gt/m/n}{}
   \DeclareFontShape{JY1}{gt}{m}{sl}{<->ssub*gt/m/n}{}
   \DeclareFontShape{JY1}{mc}{bx}{it}{<->ssub*gt/m/n}{}
   \DeclareFontShape{JY1}{mc}{bx}{sl}{<->ssub*gt/m/n}{}
   \DeclareFontShape{JT1}{mc}{m}{it}{<->ssub*mc/m/n}{}
   \DeclareFontShape{JT1}{mc}{m}{sl}{<->ssub*mc/m/n}{}
   \DeclareFontShape{JT1}{mc}{m}{sc}{<->ssub*mc/m/n}{}
   \DeclareFontShape{JT1}{gt}{m}{it}{<->ssub*gt/m/n}{}
   \DeclareFontShape{JT1}{gt}{m}{sl}{<->ssub*gt/m/n}{}
   \DeclareFontShape{JT1}{mc}{bx}{it}{<->ssub*gt/m/n}{}
   \DeclareFontShape{JT1}{mc}{bx}{sl}{<->ssub*gt/m/n}{}
   \DeclareRelationFont{JY1}{mc}{it}{}{\encodingdefault}{\rmdefault}{it}{}
   \DeclareRelationFont{JT1}{mc}{it}{}{\encodingdefault}{\rmdefault}{it}{}
   \DeclareRelationFont{JY1}{gt}{it}{}{\encodingdefault}{\rmdefault}{bx}{it}
   \DeclareRelationFont{JT1}{gt}{it}{}{\encodingdefault}{\rmdefault}{bx}{it}
   % \CLASSOPTIONnofonttunefalse
\fi

% subcaption のために caption のセットアップを行う.(conference style)
\usepackage[compatibility=false,aboveskip=.3em,belowskip=0pt]{caption}
\DeclareCaptionLabelSeparator{ieeefig}{.\nobreakspace\nobreakspace}
\captionsetup[table]{font=footnotesize,textfont=sc,justification=centering,labelsep=newline}
\captionsetup[figure]{font=footnotesize,labelsep=ieeefig}
\usepackage[font=footnotesize,labelformat=simple,aboveskip=.5em,belowskip=.8em]{subcaption}
\captionsetup[subtable]{textfont=md}
\renewcommand{\thesubfigure}{(\alph{subfigure})}
\renewcommand{\thesubtable}{(\alph{subtable})}


% その他役に立つパッケージ
\usepackage{array}
\usepackage{balance}
\usepackage{cite}
\usepackage{algorithm,algorithmic,url}
\usepackage{lipsum}

\bibliographystyle{IEEEtran}

% 数学コマンドの補完
\DeclareMathOperator{\sinc}{sinc}
\DeclareMathOperator{\prox}{prox}
\DeclareMathOperator*{\argmin}{argmin}
\DeclareMathOperator*{\argmax}{argmax}

% ハイフネーションが変な所があれば
\hyphenation{op-tical net-works semi-conduc-tor}

\definecolor{red}{rgb}{.75,0,0}

% Fig. の後のスペースが広がらないように frenchspacing にする
\frenchspacing