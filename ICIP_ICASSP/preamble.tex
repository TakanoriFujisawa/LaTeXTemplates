%!TEX root = manuscript.tex
%%% ICIP/ICASSP 向け LaTeX テンプレート (プリアンブル)

% フォント関連
\usepackage[T1]{fontenc}
\usepackage{lmodern,newtxtext}
\normalfont % fd ファイルを読み込ませる
\DeclareFontShape{T1}{\rmdefault}{m}{sc}{<-> ptmrc8t}{}
\DeclareFontShape{T1}{\rmdefault}{b}{sc}{<-> ptmbc8t}{}
\DeclareFontShape{T1}{\rmdefault}{bx}{sc}{<-> ptmbc8t}{}

\usepackage{amsmath,amssymb,amsfonts,bm,mathtools}
\usepackage{xspace,balance,cite,array,makecell}
\usepackage{algorithm,algorithmic,url,lipsum}

% 日本語が入らないようにする
\usepackage[ascii]{inputenc}

\ifx\pfmtname\undefined
   \usepackage{graphicx,color}
   \usepackage[table]{xcolor}
\else
   \usepackage[dvipdfmx]{graphicx,color}
   \usepackage[dvipdfmx,table]{xcolor}
   \DeclareFontShape{JY1}{mc}{m}{n}{<-> s*[0.961] jis}{}
   \DeclareFontShape{JY1}{gt}{m}{n}{<-> s*[0.961] jisg}{}
   \DeclareFontShape{JT1}{mc}{m}{n}{<-> s*[0.961] tmin10}{}
   \DeclareFontShape{JT1}{gt}{m}{n}{<-> s*[0.961] tgoth10}{}
   \DeclareFontShape{JY1}{mc}{m}{it}{<->ssub*mc/m/n}{}
   \DeclareFontShape{JY1}{mc}{m}{sl}{<->ssub*mc/m/n}{}
   \DeclareFontShape{JY1}{mc}{m}{sc}{<->ssub*mc/m/n}{}
   \DeclareFontShape{JY1}{gt}{m}{it}{<->ssub*gt/m/n}{}
   \DeclareFontShape{JY1}{gt}{m}{sl}{<->ssub*gt/m/n}{}
   \DeclareFontShape{JY1}{mc}{bx}{it}{<->ssub*gt/m/n}{}
   \DeclareFontShape{JY1}{mc}{bx}{sl}{<->ssub*gt/m/n}{}
   \DeclareFontShape{JT1}{mc}{m}{it}{<->ssub*mc/m/n}{}
   \DeclareFontShape{JT1}{mc}{m}{sl}{<->ssub*mc/m/n}{}
   \DeclareFontShape{JT1}{mc}{m}{sc}{<->ssub*mc/m/n}{}
   \DeclareFontShape{JT1}{gt}{m}{it}{<->ssub*gt/m/n}{}
   \DeclareFontShape{JT1}{gt}{m}{sl}{<->ssub*gt/m/n}{}
   \DeclareFontShape{JT1}{mc}{bx}{it}{<->ssub*gt/m/n}{}
   \DeclareFontShape{JT1}{mc}{bx}{sl}{<->ssub*gt/m/n}{}
   \DeclareRelationFont{JY1}{mc}{it}{}{\encodingdefault}{\rmdefault}{it}{}
   \DeclareRelationFont{JT1}{mc}{it}{}{\encodingdefault}{\rmdefault}{it}{}
   \DeclareRelationFont{JY1}{gt}{it}{}{\encodingdefault}{\rmdefault}{bx}{it}
   \DeclareRelationFont{JT1}{gt}{it}{}{\encodingdefault}{\rmdefault}{bx}{it}
\fi

% subcaption のために caption のセットアップを行う.(conference style)
\usepackage[compatibility=false]{caption}
\captionsetup[table]{font=footnotesize,belowskip=0em,aboveskip=0.8em}
\captionsetup[figure]{font=footnotesize,labelsep=period,belowskip=0em,aboveskip=0.8em}
\usepackage[font=footnotesize,labelformat=simple,belowskip=0em,aboveskip=.4em]{subcaption}
\renewcommand{\thesubfigure}{(\alph{subfigure})}
\renewcommand{\thesubtable}{(\alph{subtable})}

% 数学コマンドの補完
\DeclareMathOperator{\sinc}{sinc}
\DeclareMathOperator{\prox}{prox}
\DeclareMathOperator*{\argmin}{argmin}
\DeclareMathOperator*{\argmax}{argmax}
\DeclareMathOperator*{\minimize}{minimize}

% 1 段組みに入る図の数を調整
\renewcommand{\dbltopfraction}{.8}
\renewcommand{\textfraction}{.1}
\setcounter{topnumber}{3}

\bibliographystyle{IEEEbib}
